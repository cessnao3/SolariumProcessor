\documentclass{article}
\usepackage[paper=a4paper,margin=1in]{geometry}

\usepackage{listings}
\usepackage{graphicx}

\usepackage{comment}

%opening
\title{SProc Instruction Set Architecture}
\author{Ian O'Rourke}
\date{\today}

\begin{document}

\maketitle

\section{Overview}


SProc is a general-purpose Reduced Instruction Count (RISC) Instruction Set Architecture (ISA). Each program instruction is designed to take up exactly one word of memory. In this case, each word is 16-bits.

\subsection{Registers}

There are 16 total registers present on the SProc. These are defined generally as follows in Table \ref{table:register-setup}. Note that, as SProc is a 16-bit architecture, this means that each register is 16 bits wide.

\begin{table}[h!]
	\centering
	\begin{tabular}{c|c}
		\hline
		Register & Usage \\
		\hline
		R0 & Program Counter \\
		R1 & Global Stack Pointer \\
		R2 & Return Value \\
		R3-R15 & General Purpose Register \\
		\hline
	\end{tabular}
	\caption{Outside of the program counter and global stack pointer, the registers within the SProc are all general-purpose.}
	\label{table:register-setup}
\end{table}

\subsection{Overall Instruction Syntax}

Since the architecture of the SProc is 16-bit, this means that each word that may be addressed is 16-bits. The general instruction format is listed  in Table \ref{table:instruction-formatting}.

\begin{table}[h!]
	\centering
	\begin{tabular}{l|cccc}
		\hline
		Location & 0xF000 & 0x0F00 & 0x00F0 & 0x000F \\
		\hline
		Usage & opcode & arg0 & arg1 & arg2 \\
		\hline
	\end{tabular}
	\caption{The SProc instruction format typically has one opcode and three possible arguments associated with a particular opcode}
	\label{table:instruction-formatting}
\end{table}

This is contrasted with the typical assembly language formatting, which is provided as

\begin{center}
	\texttt{instruction <arg2>, <arg1>, <arg0>}
\end{center}

where the number of arguments depends on the required number of arguments for the instruction.

\pagebreak

\section{Available Instructions}

All available instructions are listed in Table \ref{table:instruction-table}. Note that any invalid instruction that is not provided in the table below results in an immediate halt of the processor.

\begin{table}[h!]
	\centering
	\begin{footnotesize}
		\begin{tabular}{cccc|c|l}
			\hline
			opcode & arg0 & arg1 & arg2 & Assembly & Description \\
			\hline
			0 & 0 & 0 & 0 & \texttt{noop} & No Operation \\
			0 & 0 & 0 & 1 & \texttt{inton} & Turn Interrupts On \\
			0 & 0 & 0 & 2 & \texttt{intoff} & Turn Interrupts Off \\
			0 & 0 & 0 & 3 & \texttt{reset} & \texttt{PC = Reset Vector}, \texttt{R[0-15] = 0} \\
			0 & 0 & 0 & 4 & \texttt{pop} & \texttt{--SP} \\
			0 & 0 & 0 & 5 & \texttt{ret} & $\forall_{i \in [15 \rightarrow 0]}$ \texttt{R[i] = mem[--SP]}, \texttt{++PC} \\
			0 & 0 & 1 & R[a] & \texttt{jmp [a]} & \texttt{PC = R[a]} \\
			0 & 0 & 2 & R[a] & \texttt{jmpr [a]} & \texttt{PC += R[a]} \\
			0 & 0 & 3 & R[a] & \texttt{push [a]} & \texttt{mem[SP++] = R[a]} \\
			0 & 0 & 4 & R[a] & \texttt{popr [a]} & \texttt{R[a] = mem[--SP]} \\
			0 & 0 & 5 & R[a] & \texttt{call [a]} & $\forall_{i \in [0 \rightarrow 15]}$
			 \texttt{mem[SP++] = R[i]}, \texttt{PC = R[a]} \\
 			0 & 0 & 6 & [a] & \texttt{int R[a]} & Trigger the Software Interrupt within \texttt{R[a]} \\
			0 & 1 & 0xI0 & 0x0I & \texttt{jmpri <imm>} & \texttt{PC += Im} \\
			0 & 2 & R[src] & R[dst] & \texttt{ld [dst], [src]} & \texttt{R[dst] = mem[R[src]]} \\
			0 & 3 & R[src] & R[dst] & \texttt{sav [dst], [src]} & \texttt{mem[R[dst]] = R[src]} \\
			0 & 4 & R[src] & R[dst] & \texttt{ldr [dst], [src]} & \texttt{R[dst] = mem[PC + R[src]]} \\
			0 & 5 & R[src] & R[dst] & \texttt{savr [dst], [src]} & \texttt{mem[PC + R[dst]] = R[src]} \\
			0 & 6 & R[cmp] & R[a] & \texttt{jz [a], [cmp]} & \texttt{PC = R[a] IF R[cmp] == 0} \\
			0 & 7 & R[cmp] & R[a] & \texttt{jzr [a], [cmp]} & \texttt{PC += R[a] IF R[cmp] == 0} \\
			0 & 8 & R[cmp] & R[a] & \texttt{jgz [a], [cmp]} & \texttt{PC = R[a] IF R[cmp] > 0} \\
			0 & 9 & R[cmp] & R[a] & \texttt{jgzr [a], [cmp]} & \texttt{PC += R[a] IF R[cmp] > 0} \\
			1 & 0xI0 & 0x0I & R[dst] & \texttt{ldi [dst], <imm>} & \texttt{R[dst] = Im} \\
			2 & 0xI0 & 0x0I & R[dst] & \texttt{ldui [dst], <imm-unsigned>} & \texttt{R[dst] = Im} \\
			3 & 0xI0 & 0x0I & R[dst] & \texttt{ldir [dst], <imm>} & \texttt{R[dst] = mem[PC + Im]} \\
			4 & R[b] & R[a] & R[dst] & \texttt{add [dst], [a], [b]} & \texttt{R[dst] = R[a] + R[b]} \\
			5 & R[b] & R[a] & R[dst] & \texttt{sub [dst], [a], [b]} & \texttt{R[dst] = R[a] - R[b]} \\
			6 & R[b] & R[a] & R[dst] & \texttt{mul [dst], [a], [b]} & \texttt{R[dst] = R[a] * R[b]} \\
			7 & R[b] & R[a] & R[dst] & \texttt{div [dst], [a], [b]} & \texttt{R[dst] = R[a] / R[b]} \\
			8 & R[b] & R[a] & R[dst] & \texttt{mod [dst], [a], [b]} & \texttt{R[dst] = R[a] \% R[b]} \\
			9 & R[b] & R[a] & R[dst] & \texttt{band [dst], [a], [b]} & \texttt{R[dst] = R[a] \& R[b]} \\
			10 & R[b] & R[a] & R[dst] & \texttt{bor [dst], [a], [b]} & \texttt{R[dst] = R[a] | R[b]} \\
			11 & R[b] & R[a] & R[dst] & \texttt{bxor [dst], [a], [b]} & \texttt{R[dst] = R[a] $\wedge$ R[b]} \\
			12 & R[b] & R[a] & R[dst] & \texttt{bsftl [dst], [a], [b]} & \texttt{R[dst] = R[a] << R[b]} \\
			13 & R[b] & R[a] & R[dst] & \texttt{bsftr [dst], [a], [b]} & \texttt{R[dst] = R[a] >> R[b]} \\
			\hline
		\end{tabular}
	\end{footnotesize}
	\caption{Available instruction list for the SProc provides a variety of commands.}
	\label{table:instruction-table}
\end{table}

\pagebreak

\section{Examples}

The following list some simple example programs that can be run on the SProc.

\subsection{Counter}

The program listed in Listing \ref{listing:example-counter} provides a basic counter. A target value is placed in register four, and the value in register three is incremented from 0 to the target value in register four by adding one to the register each loop. Once the target value has been reached and register three is equal to register four, the program halts by entering an infinite loop.

\lstinputlisting[caption={Counter Program Example}, label={listing:example-counter}]{../examples/counter.smc}

\pagebreak

\section{Tools}

Several tools can help in the development of SProc programs.

\subsection{VisualSProc}

One useful tool is VisualSProc, which is a program that combines together a basic assembler, CPU emulator, and memory inspector into a single program. The main window can be seen in Figure \ref{fig:visual-sproc-main-page}.

\begin{figure}[h!]
	\centering
	\includegraphics[width=5in]{images/visual-sproc.png}
	\caption{Main window of VisualSProc provides common tools for program writing}
	\label{fig:visual-sproc-main-page}
\end{figure}


\end{document}
