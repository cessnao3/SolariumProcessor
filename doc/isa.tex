\documentclass{article}
\usepackage[paper=a4paper,margin=1in]{geometry}

\usepackage{comment}

%opening
\title{SCPU Instruction Set Architecture}
\author{Ian}
\date{\today}

\begin{document}

\maketitle

\section{Overview}


The SCPU is a general-purpose Reduced Instruction Count (RISC) Instruction Set Architecture (ISA). Each program instruction is designed to take up exactly one word of memory. In this case, each word is 16-bits.

\subsection{Registers}

There are 16 total registers present on the SCPU. These are defined generally as follows in Table \ref{table:register-setup}. Note that, as SCPU is a 16-bit architecture, this means that each register is 16 bits wide.

\begin{table}[h!]
	\centering
	\begin{tabular}{c|c}
		\hline
		Register & Usage \\
		\hline
		R0 & Program Counter \\
		R1 & Global Stack Pointer \\
		R2-R15 & General Purpose Register \\
		\hline
	\end{tabular}
	\caption{Outside of the program counter and global stack pointer, the registers within the SCPU are all general-purpose.}
	\label{table:register-setup}
\end{table}

\subsection{Overall Instruction Syntax}

Since the architecture of the SCPU is 16-bit, this means that each word that may be addressed is 16-bits.

\begin{table}[h!]
	\centering
	\begin{tabular}{l|cccc}
		\hline
		Location & 0xF000 & 0x0F00 & 0x00F0 & 0x000F \\
		\hline
		Usage & opcode & arg0 & arg1 & arg2 \\
		\hline
	\end{tabular}
	\caption{The SCPU instruction format typically has one opcode and three possible arguments associated with a particular opcode}
	\label{table:instruction-formatting}
\end{table}

\pagebreak

\section{Available Instructions}

\begin{table}[h!]
	\centering
	\begin{footnotesize}
		\begin{tabular}{cccc|c|l}
			\hline
			opcode & arg0 & arg1 & arg2 & Assembly & Description \\
			\hline
			0 & 0 & 0 & 0 & \texttt{noop} & No Operation \\
			0 & 0 & 0 & 1 & \texttt{inton} & Interrupts On \\
			0 & 0 & 0 & 2 & \texttt{intoff} & Interrupts Off \\
			0 & 0 & 0 & 3 & \texttt{int} & Call An Interrupt \\
			0 & 0 & 0 & 4 & \texttt{reset} & Resets the Processor \\
			0 & 0 & 0 & 5 & \texttt{pop} & \texttt{SP--} \\
			0 & 0 & 0 & 6 & \texttt{ret} & Returns from a previous function call \\
			0 & 0 & 1 & R[a] & \texttt{jmp R[a]} & \texttt{PC = R[a]} \\
			0 & 0 & 2 & R[a] & \texttt{jmpr R[a]} & \texttt{PC += R[a]} \\
			0 & 0 & 3 & R[a] & \texttt{push R[a]} & \texttt{mem[SP++] = R[a]} \\
			0 & 0 & 4 & R[a] & \texttt{call R[a]} & Calls the function/routine at the location Ra \\
			0 & 1 & 0xI0 & 0x0I & \texttt{jmpri 0xII} & \texttt{PC += Im} \\
			0 & 2 & R[src] & R[dst] & \texttt{ld Rdst, Rsrc} & \texttt{R[dst] = mem[R[src]]} \\
			0 & 3 & R[src] & R[dst] & \texttt{sav Rdst, Rsrc} & \texttt{mem[R[dst]] = R[src]} \\
			0 & 4 & R[src] & R[dst] & \texttt{ldr Rds, Rsrc} & \texttt{R[dst] = mem[PC + R[src]]} \\
			0 & 5 & R[src] & R[dst] & \texttt{svr Rds, Rdst} & \texttt{mem[PC + R[dst]] = R[src]} \\
			0 & 6 & R[cmp] & R[a] & \texttt{jz Rcmp, Ra} & \texttt{PC = R[a] IF R[cmp] == 0} \\
			0 & 7 & R[cmp] & R[a] & \texttt{jzr Rcmp, Ra} & \texttt{PC += R[a] IF R[cmp] == 0} \\
			0 & 8 & R[cmp] & R[a] & \texttt{jgz Rcmp, Ra} & \texttt{PC = R[a] IF R[cmp] > 0} \\
			0 & 9 & R[cmp] & R[a] & \texttt{jgzr Rcmp, Ra} & \texttt{PC += R[a] IF R[cmp] > 0} \\
			1 & 0xI0 & 0x0I & R[dst] & \texttt{ldi 0xII} & \texttt{R[dst] = 0xII} \\
			2 & R[a] & R[b] & R[dst] & \texttt{add Rdst, Ra, Rb} & \texttt{R[dst] = R[a] + R[b]} \\
			3 & R[a] & R[b] & R[dst] & \texttt{sub Rdst, Ra, Rb} & \texttt{R[dst] = R[a] - R[b]} \\
			4 & R[a] & R[b] & R[dst] & \texttt{mul Rdst, Ra, Rb} & \texttt{R[dst] = R[a] * R[b]} \\
			5 & R[a] & R[b] & R[dst] & \texttt{div Rdst, Ra, Rb} & \texttt{R[dst] = R[a] / R[b]} \\
			6 & R[a] & R[b] & R[dst] & \texttt{mod Rdst, Ra, Rb} & \texttt{R[dst] = R[a] \% R[b]} \\
			7 & R[a] & R[b] & R[dst] & \texttt{band Rdst, Ra, Rb} & \texttt{R[dst] = R[a] \& R[b]} \\
			8 & R[a] & R[b] & R[dst] & \texttt{bor Rdst, Ra, Rb} & \texttt{R[dst] = R[a] - R[b]} \\
			9 & R[a] & R[b] & R[dst] & \texttt{bsft Rdst, Ra, Rb} & \texttt{R[dst] = R[a] << R[b]} \\
			\hline
		\end{tabular}
	\end{footnotesize}
	\label{table:instruction-table}
	\caption{Available instruction list for the SCPU provides a variety of commands.}
\end{table}

\end{document}
